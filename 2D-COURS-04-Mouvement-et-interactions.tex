
\chapter{Mouvement et interactions}
\intro{La \textit{dynamique} est l'étude de la \textit{modification du mouvement} d'un
objet du fait de \textit{l'interaction de ce dernier avec le reste de l'Univers}. \\
Les \textit{interactions} seront décrites par des \textit{forces}, l'étude du 
mouvement s'appelle la \textit{cinématique}, on y décrit la \textit{trajectoire de l'objet}
ainsi que l'évolution de sa vitesse.}

%%%%%%%%%%%%%%%%%%%%%%%%%%%%%%%%%%%%%%%%%%%%%%%%%%%%%%%%%%%%%%%%%%%%%
\section{Décrire un mouvement}
\subsection{Exemples de mouvements}
Les types d'objets et de mouvements étudiés en physique sont très variés
\begin{itemize}
 \item la Terre décrit une ellipse d'environ $150$ millions de $km$ de rayon en
 $1$ an autour du Soleil
 \item une plaque tectonique se déplace par rapport à une autre de quelques centimètres 
 par an.
 \item un électron d'un microscope électronique se déplace à plusieurs milliers de kilomètre
 par seconde
 \item une molécule dans un gaz à température ambiante se déplace à quelques centaines 
 de mètre par seconde
 \item un humain en marche normale se déplace à $4~km.h^{-1}$
 \item la station spatiale internationale ISS est en orbite basse à $400~km$ d'altitude et fait 
 le tour de la Terre en $1h~30min$
\end{itemize}

\subsection{Système et référentiel}
\paragraph{Définition}Un \textit{système} est l'objet (ou le groupe d'objet) dont on va décrire le mouvement. 
Il \textit{subit} l'influence du \textit{reste de l'Univers}.  
\paragraph{Définition}Pour \textit{décrire un mouvement}, on doit pouvoir \textit{mesurer la position du système à chaque instant}.
On doit donc \textit{choisir un autre objet de référence} par rapport au quel on pourra mesurer la position de notre système.
Mathématiquement, on choisira un \textit{repère orthonormé associé à une origine fixé sur le référentiel}. Voir figure \ref{fig:def_referentiel}.
\begin{figure}[h!]
  \begin{center}
      \includegraphics[width=\columnwidth]{4.0-mvt-interactions/def_referentiel.pdf}
  \end{center}
  \caption{Le système étudié est une fusée au décollage. Son mouvement est mesuré par rapport au sol en fonction du temps. On peut alors définir des coordonnés $(x,y,z)$ de la fusée à chaque instant $t$.}
  \label{fig:def_referentiel}
\end{figure}


\subsection{Relativité du mouvement} 
\paragraph{Définition} Le \textit{choix du référentiel est important}, car il conditionne la description du mouvement. Le mouvement est relatif au référentiel choisi. 
\paragraph{Exemples} 
\begin{itemize}
 \item Quand on étudie le mouvement d'un passager d'un train, il semble immobile \textit{par rapport au fauteuil} mais être en déplacement \textit{par rapport à un arbre à coté de la voie}. 
 \item Dans un film, on utilise cet effet lors d'un \textit{traveling} où la caméra suit les acteurs dans la scène en roulant sur des rails ou à bord d'un véhicule.
\end{itemize}

\subsection{Modèle du point matériel}
\paragraph{Définition} On utilise un \textit{ modèle simplifié d'un objet} dont on veut décrire le mouvement. L'objet se résume à \textit{un point où toute sa masse se trouve concentrée}. Ce point s'appelle le \textit{centre d'inertie CI}.
La position de ce point dans l'espace nécessite trois coordonnés $CI~ \left( x,y,z \right)$. \\ 
Ce modèle a ses limites, on a perdu toute information concernant l'orientation de l'objet dans l'espace et toute information concernant les points d'applications des forces. Voir figure \ref{fig:modele_point_materiel}.
\begin{figure}[h!]
  \begin{center}
      \includegraphics[width=\columnwidth]{4.0-mvt-interactions/modele_point_materiel.pdf}
  \end{center}
  \caption{Une capsule Apollo chute dans l'atmosphère \textit{(a)}, elle subit une force d'attraction
  vers la Terre et des forces de frottement sur son bouclier thermique \textit{(b)}, on simplifie
  la description avec le modèle du point matériel, où la capsule se résume à une \textit{masse ponctuelle} de $5900~kg$ soumise à deux forces.}
  \label{fig:modele_point_materiel}
\end{figure}

\paragraph{Définition} Les positions successives dans l'espace du centre d'inertie de l'objet décrit une courbe appelée
une trajectoire. On y indique des positions $(x,y,z)$ à des dates précises $t$. Voir figure \ref{fig:def_trajectoire}.
\begin{figure}[h!]
  \begin{center}
      \includegraphics[width=\columnwidth]{4.0-mvt-interactions/trajectoire_point.pdf}
  \end{center}
  \caption{La trajectoire d'un point matériel est la courbe définie par les positions successives de l'objet au cours de son mouvement décrit dans un référentiel}
  \label{fig:def_trajectoire}
\end{figure}
\paragraph{Exemples de trajectoires}
\begin{itemize}
 \item trajectoire rectiligne  
 \item trajectoire circulaire
 \item trajectoire parabolique
\end{itemize}


\subsection{Vecteur déplacement}
\paragraph{Définition} Soient deux points consécutifs $M_1(x_1,y_1)$ et $M_2(x_2,y_2)$  sur une trajectoire à des instants voisins séparés d'une durée $\Delta t$ . On appelle \textit{vecteur déplacement} le vecteur 
\begin{equation*}
  \begin{aligned}
    \overrightarrow{M_1M_2} & = (x_2-x_1) \overrightarrow i + (y_2-y_1) \overrightarrow j \\
			    & = \Delta x \overrightarrow i + \Delta y  \overrightarrow j
 \end{aligned} 
\end{equation*}
Voir figure \ref{fig:vecteur_deplacement}.
\begin{figure}[h!]
  \begin{center}
      \includegraphics[width=\columnwidth]{4.0-mvt-interactions/vecteur_deplacement.pdf}
  \end{center}
  \caption{Le vecteur déplacement permet de passer du point $M_1$ de la trajectoire au point $M_2$}
  \label{fig:vecteur_deplacement}
\end{figure}

\paragraph{Exemple} Sur la figure \ref{fig:vecteur_deplacement}, on mesure les coordonnées des points $M_1 (1.5 , 2.4)$ et $M_2 (3.5,1.7)$. On peut ensuite calculer le vecteur déplacement 
\begin{equation*}
  \begin{aligned}
    \overrightarrow{M_1M_2} &= (3.5-1.5) \overrightarrow {i} + (1.7-2.4) \overrightarrow{ j}\\
		& = 2.0 \overrightarrow{i} - 0.7 \overrightarrow{j}
\end{aligned} 
\end{equation*}
On a un déplacement de $\Delta x = 2.0~m$ vers la droite et de $\Delta y = -0.7~m$ vers le bas.

\subsection{Vecteur vitesse moyenne}
\paragraph{Définition} Le \textit{vecteur vitesse moyenne} $\overrightarrow V$ peut se calculer 
entre deux points séparés d'une durée $\Delta t$ à partir du vecteur déplacement (voir figure \ref{fig:vecteur_vitesse_moyenne}).
\begin{equation*}
  \begin{aligned}
    \overrightarrow{V}    & =  \frac{1}{\Delta t} \overrightarrow{M_1M_2} \\
			  & =  \frac{1}{\Delta t} \overrightarrow{M_1M_2} \\
			  & =  \frac{1}{\Delta t} \left( (x_2-x_1) \overrightarrow{i} + (y_2-y_1) \overrightarrow{j} \right) \\
			  & =  \frac{(x_2-x_1)}{\Delta t} \overrightarrow{i} + \frac{(y_2-y_1)}{\Delta t} \overrightarrow{j}  \\
			  & =  \frac{\Delta x}{\Delta t} \overrightarrow{i} + \frac{\Delta y}{\Delta t} \overrightarrow{j}  \\
			  & =  V_x \overrightarrow{i} + V_y \overrightarrow{j} \\
  \end{aligned} 
\end{equation*}
avec $V_x = \dfrac{\Delta x}{\Delta t}$ la vitesse selon l'axe $Ox$ et $V_y = \dfrac{\Delta y}{\Delta t}$ la vitesse selon l'axe $Oy$.

\begin{figure}[h!]
  \begin{center}
      \includegraphics[width=\columnwidth]{4.0-mvt-interactions/vecteur_vitesse_moyenne.pdf}
  \end{center}
  \caption{Le vecteur vecteur vitesse moyenne se calcule à partir du vecteur déplacement entre deux points de la trajectoire séparés par une durée $\Delta t$}
  \label{fig:vecteur_vitesse_moyenne}
\end{figure}

\paragraph{Exemple} Sur la figure \ref{fig:vecteur_vitesse_moyenne}, on peut mesurer le vecteur déplacement  $\overrightarrow{M_1M_2} = 2.0 \overrightarrow{i} - 0.7 \overrightarrow {j} $ et on connaît la durée qui sépare deux positions successives sur la trajectoire $\Delta t =  50~ms = 50 \times 10^{-3}~s$. On peut alors calculer les coordonnées du vecteur vitesse moyenne 
$$V_x = \dfrac{\Delta x}{\Delta t} = \dfrac{2.0}{50 \times 10^{-3}~s} = 40~m.s^{-1} $$
$$V_y = \dfrac{\Delta y}{\Delta t} = \dfrac{-0.7}{50 \times 10^{-3}~s} = -14~m.s^{-1} $$
Donc 
$$\overrightarrow{V} = 40 \overrightarrow{i} - 14 \overrightarrow{j}$$
et $$ V = \sqrt{V_x^2 + V_y^2} = \sqrt{(40)^2 + (-14)^2} = 43 m.s^{-1}$$


\subsection{Vecteur vitesse en un point}
\paragraph{Définition} Le \textit{vecteur vitesse en un point $\overrightarrow{v_M}$} a pour norme la vitesse moyenne entre le point $M$ et le point suivant $M'$ c 'est à dire le rapport entre la distance parcourue de $M$ à $M'$ et la durée de ce parcours $\Delta  t$ 
$$ \lVert \overrightarrow{v_M} \rVert = \frac{MM'}{\Delta t} $$ 
$\overrightarrow{v_M}$ a pour sens le sens du mouvement et a pour direction la tangente à la trajectoire au point $M$.

\paragraph{Remarque} Cette méthode est imprécise, si la vitesse varie beaucoup sur l'intervalle $MM'$. Il existe d'autres méthodes d'estimation de la vitesse au point $M$ plus précises mais plus complexes à mettre en œuvre.
\begin{figure}[h!]
  \begin{center}
      \includegraphics[width=\columnwidth]{4.0-mvt-interactions/vecteur_vitesse_en_un_point.pdf}
  \end{center}
  \caption{Le vecteur vitesse au point $M_1$ se calcule à partir de la vitesse moyenne entre les points $M_1$ et $M_2$ et se dessine au point $M_1$ tangent à la trajectoire, dans le sens du mouvement}
  \label{fig:vec_vitesse_en_un_point}
\end{figure}
\paragraph{Exemple} Sur la figure \ref{fig:vec_vitesse_en_un_point}, on peut mesurer les coordonnées des points $M_1(1.5,2.4)$ et $M_2(3.5,1.7)$ ainsi que l'intervalle de temps entre deux positions successives sur la trajectoire $\Delta t = 50~ms = 50 \times 10^{-3}~s$. On peut alors calculer la norme du vecteur vitesse au point $M_1$.
\begin{equation*}
  \begin{aligned}
   	\lVert \overrightarrow{v_{M_1}} \rVert & = \frac{M_1M_2}{\Delta t} \\
	      & = \frac{ \sqrt{(3.5-1.5)^2 + (1.7 - 2.4)^2}~m}{50 \times 10^{-3} s  } \\
	      & =42~m.s^{-1}
  \end{aligned} 
\end{equation*}
La vitesse au point $M_1$ a pour norme $42~m.s^{-1}$, est tangente à la trajectoire au point $M_1$ et est orientée dans le sens du mouvement, vers $M_2$.


\subsection{Mouvement rectiligne }
\paragraph{Définition} Un mouvement est dit \textit{rectiligne uniforme} si le \textit{vecteur vitesse en un point} est constant dans le temps.
$$ V = \textit{constant} $$
$$ x = V \times t + x_0 $$
Un mouvement est dit \textit{rectiligne non uniforme} si le \textit{vecteur vitesse en un point} varie dans le temps en norme ou en sens mais pas en direction. Par exemple
$$ V = a \times t  + V_0 $$
$$ x = \frac{1}{2}\times a \times t^2 + V_0 \times t + x_0 $$
Voir figure \ref{fig:mvt_rec_unif}.
\begin{figure}[h!]
  \begin{center}
      \includegraphics[width=\columnwidth]{4.0-mvt-interactions/mvt_rect_unif_non_unif.pdf}
  \end{center}
  \caption{La figure $a)$ représente un mouvement rectiligne et uniforme, l'objet se déplace en ligne droite à vitesse constante. La figure $b)$ représente un mouvement rectiligne non uniforme, l'objet se déplace en ligne droite mais sa vitesse varie, ici, elle augmente, l'objet accélère.}
  \label{fig:mvt_rec_unif}
\end{figure}

\paragraph{Exemples}
\begin{itemize}
 \item une glissade sur une surface horizontale est un mouvement rectiligne uniforme
 \item une chute libre sans frottement est un mouvement rectiligne accélérée
 \item une chute sous un parachute est un mouvement rectiligne uniforme
 \item une bille en acier qui tombe dans la neige a un mouvement rectiligne non uniforme (décélération) 
\end{itemize}

%%%%%%%%%%%%%%%%%%%%%%%%%%%%%%%%%%%%%%%%%%%%%%%%%%%%%%%%%%%%%%%%%%%%%
\section{Modélisation d'une action sur un système}
\subsection{Action sur un système}
\paragraph{Définition} On définit en premier le \textit{système étudié}. Il va \textit{subir} l'action d'un autre système. Les systèmes sont des objets dont on veut étudier la dynamique, c'est à dire la façon dont leur mouvement va changer sous l'action d'autres objets. Voir figure \ref{fig:system_doi}.
\begin{figure}[h!]
  \begin{center}
      \includegraphics[width=\columnwidth]{4.0-mvt-interactions/systeme_doi_action.pdf}
  \end{center}
  \caption{Le système étudié va subir des actions de la part des autres objets de l'Univers.}
  \label{fig:system_doi}
\end{figure}

\paragraph{Définition} Une \textit{action sur un système} va être modélisée mathématiquement par un
\textit{vecteur force} $\overrightarrow{F}$ dont on doit préciser
\begin{itemize}
 \item la direction
 \item le sens
 \item la norme $F$ qui s'exprimera en Newton $N$
\end{itemize}
Comme nous considérons que le système se résume à un \textit{point matériel}, le point d'application
de la force $\overrightarrow{F}$ sera le point.
\paragraph{Exemple} 
Le \textit{système étudié} est une boite de masse $m=500~g$ posée sur une table, elle subit l'action de la Terre modélisée par le vecteur force poids $\overrightarrow {P}$ et l'action de la table modélisée par le vecteur réaction de la table $\overrightarrow {R}$. 
Voir figure \ref{fig:exemple_vec_force}.
\begin{figure}[h!]
  \begin{center}
      \includegraphics[width=\columnwidth]{4.0-mvt-interactions/exemple_vec_force.pdf}
  \end{center}
  \caption{Le système étudié est une boite posée sur une table et sous l'influence de la Terre (figures $a$ et $b$). On modélise la boite par un objet ponctuel soumis aux forces $\protect\overrightarrow{P}$ et $\protect\overrightarrow{R}$ (figure $c$).}
  \label{fig:exemple_vec_force}
\end{figure}

\paragraph{Méthode} 
Pour modéliser les actions, il faut suivre les étapes suivantes
\begin{enumerate}
 \item Définir précisément le système étudié qui va subir de la part de l'extérieur des actions qui seront modélisées par des vecteurs forces
 \item Faire l'inventaire de l'ensemble des actions extérieures appliquées au système étudié
 \item Pour chaque action, définir précisément le vecteur force, c'est à dire qu'il faut donner son sens, sa direction et sa norme, la norme étant exprimée en Newton ($N$)
 \item Sur un schéma simplifié du système étudié (modèle du point matériel), dessiner précisément l'ensemble des forces, en respectant leur sens, leur direction et leur norme, on indiquera alors une échelle pour dessiner des vecteurs forces.
\end{enumerate}


\subsection{Principe des actions réciproques - 3\ieme{} loi de Newton}
\paragraph{Définition} On étudie un système $A$ qui subit une force $\overrightarrow{F_{B/A}}$
de la part d'un système $B$. Si on considère le système $B$, il subit de la part du système $A$ une
force $\overrightarrow{F_{A/B}}$. Le \textit{principe des actions réciproque} dit alors que 
$$\overrightarrow{F_{B/A}} =  -\overrightarrow{F_{A/B}} $$
  Voir figure \ref{fig:def_actions_reciproques}.

\begin{figure}[h!]
  \begin{center}
      \includegraphics[width=\columnwidth]{4.0-mvt-interactions/def_actions_reciproques.pdf}
  \end{center}
  \caption{Si le système étudié est l'objet B alors l'objet A exerce une force sur l'objet B $\protect\overrightarrow{F_{A/B}}$. Si le système étudié est l'objet A alors l'objet B exerce une force sur l'objet A  $\protect\overrightarrow{F_{B/A}}$. D'après le principe des actions réciproques $\protect\overrightarrow{F_{B/A}}= - \protect\overrightarrow{F_{A/A}}$  }
  \label{fig:def_actions_reciproques}
\end{figure}



\subsection{Exemples d'interactions de contact}
\paragraph{Définition} Une \textit{interaction de contact} nécessite que les deux systèmes soient en contact.
\paragraph{Exemple} 
\begin{itemize}
 \item Une brique est suspendue à un fil. Elle subit de la part du fil une force de traction $\overrightarrow{T}$ dirigée le long du fil, orientée vers le fil. 
 \item Une brique est posée sur une table. Elle subit de la part de la surface une force de réaction $\overrightarrow{R}$ dirigée perpendiculairement à la surface, orientée vers l'objet posé. 
\end{itemize}
Voir figure \ref{fig:interaction_contact}.
\begin{figure}[h!]
  \begin{center}
      \includegraphics[width=\columnwidth]{4.0-mvt-interactions/interaction_contact.pdf}
  \end{center}
  \caption{Les interactions de contact nécessitent que les objets interagissants se touchent. }
  \label{fig:interaction_contact}
\end{figure}

\subsection{Exemples d'interactions à distance}
\paragraph{Définition} Une \textit{interaction à distance} ne nécessite pas que les deux systèmes soient en contact.

\paragraph{Définition} Un objet de masse $m$ placé à proximité de la surface d'une planète subit une force $\overrightarrow{P}$ appelée
\textit{le poids} qui est une force verticale, dirigée vers le centre de la planète. La norme du poids se calcule par la
relation $$P = m \times g$$ avec 
\begin{itemize}
 \item $P$ en Newton $N$
 \item $m$ en kilogramme $kg$
 \item $g$ accélération de pesanteur, dépend de la planète, en $N.kg^{-1}$
\end{itemize}
Voir figure \ref{fig:interaction_poids}.
\begin{figure}[h!]
  \begin{center}
      \includegraphics[width=\columnwidth]{4.0-mvt-interactions/interaction_poids.pdf}
  \end{center}
  \caption{Le poids d'un objet de masse $m$ est une force d'interaction à distance que l'objet subit à proximité de la surface d'une planète. }
  \label{fig:interaction_poids}
\end{figure}

\paragraph{Exemple} Un objet possède une masse $m=500~g$. On peut calculer son poids à la surface de différentes planètes connaissant la valeur de l'accélération de pesanteur $g$ sur ces planètes. Il faudra faire attention lors du calcul aux unités à respecter et donc convertir la masse de l'objet en kilogrammes. Voir tableau \ref{tab:calcul_poids}.
\begin{table}[h!]
  \centering
  \begin{tabu} to 0.95\linewidth {  X[l]  X[c] X[c]  }
    \hline
     \textbf{Planète} & \textbf{g} ($N.kg^{-1}$) & \textbf{P}  ($N$)\\
    \hline
      Terre  & $9.81$ & $4.9$ \\
      Mars  & $3.7$ & $1.9$ \\
      Lune  & $1.6$ & $0.8$ \\
      Comète & $5 \times 10^{-4}$ & $2.5\times 10^{-4}$ \\
    \hline
  \end{tabu}
  \caption{Valeurs du poids d'un objet de $500~g$ à la surface de différents astres }
  \label{tab:calcul_poids}
\end{table}
\begin{figure}[h!]
  \begin{center}
      \includegraphics[width=\columnwidth]{4.0-mvt-interactions/interaction_poids.pdf}
  \end{center}
  \caption{Le poids d'un objet de masse $m$ est une force d'interaction à distance que l'objet subit à proximité de la surface d'une planète. }
  \label{fig:interaction_poids}
\end{figure}

\paragraph{Définition} Un objet $A$ de masse $M$, placé à une distance $d$ d'un autre objet $B$ de masse $m$ subit une
\textit{force d'attraction gravitationnelle} $\overrightarrow{F_{B/A}}$ 
\begin{itemize}
 \item dont la direction est la droite passant par les centres d'inertie des deux objets
 \item orientée de l'objet de masse $m$ vers l'objet de masse $M$ (force attractive)
 \item dont la norme est donnée par la formule $$ F = \frac{G \times m \times M}{d^2}$$ 
\end{itemize}
Les unités à respecter sont
\begin{itemize}
 \item $F$ en Newton $N$
 \item $m$ et $M$ en kilogramme $kg$
 \item $d$ en mètre $m$
 \item $G=6.67 \times 10^{-11}~N.m^2.kg^{-2}$
\end{itemize}
Voir figure \ref{fig:interaction_gravitation}.
\begin{figure}[h!]
  \begin{center}
      \includegraphics[width=\columnwidth]{4.0-mvt-interactions/interaction_gravitation.pdf}
  \end{center}
  \caption{Le système étudié est un objet $A$ de masse $M$ qui subit de la part d'un objet  $B$ de masse $m$ une force d'attraction gravitationnelle $\protect\overrightarrow{F_{B/A}} $ }
  \label{fig:interaction_gravitation}
\end{figure}

\paragraph{Exemple} Soit deux objets de masse $M=10~kg$ et $m=1~kg$ disposés à $d=2~m$ l'un de l'autre. On peut calculer la valeur de la force d'attraction gravitationnelle entre les deux corps massifs 
\begin{equation*}
  \begin{aligned}
    F   &= \frac{G \times m \times M}{d^2} \\
	&= \frac{6.67 \times 10^{-11} \times 1.0 \times 10}{2.0^2} \\		
	&= 1.67 \times 10^{-10}~N 
  \end{aligned} 
\end{equation*}
Soit un satellite de masse $M=1~t$ situé à une distance $d=100~000~km$ de la Terre qui a une masse $M=5.972 \times 10^{24}~kg$. On peut calculer la valeur de la force d'attraction gravitationnelle que subit ce satellite. Pour faire ce calcul, il faudra convertir les masses en kilogramme, et les distances en mètre.
\begin{equation*}
  \begin{aligned}
    F   &= \frac{G \times m \times M}{d^2} \\
	&= \frac{6.67 \times 10^{-11} \times 5.972 \times 10^{24} \times 1000}{100000000^2} \\		
	&= 40~N 
  \end{aligned} 
\end{equation*}


%%%%%%%%%%%%%%%%%%%%%%%%%%%%%%%%%%%%%%%%%%%%%%%%%%%%%%%%%%%%%%%%%%%%%
\section{Principe d'inertie}
\subsection{Énonce} 
\paragraph{Définition} 
\begin{itemize}
 \item Si \textit{la somme des forces extérieures} que subit un objet est \textit{nulle}, alors ce corps garde un \textit{vecteur vitesse constant}.
 \item Si un corps a \textit{un vecteur vitesse qui varie}, alors cela signifie que \textit{la somme des forces extérieures} qu'il subit \textit{n'est pas nulle}.
\end{itemize}

\subsection{Cas du point immobile}
\paragraph{Définition}
Pour un point immobile
\begin{itemize}
 \item la somme des forces extérieures est nulle
 \item le vecteur vitesse reste nul $\overrightarrow{V} = \overrightarrow{0}$
\end{itemize}
Voir figure \ref{fig:principe_inertie_statique}.
\begin{figure}[h!]
  \begin{center}
      \includegraphics[width=\columnwidth]{4.0-mvt-interactions/principe_inertie_statique.pdf}
  \end{center}
  \caption{Un objet sur lequel les forces se compensent peut rester immobile. Un objet  immobile est soumis à des forces qui doivent se compenser.}
  \label{fig:principe_inertie_statique}
\end{figure}

\subsection{Cas du point en mouvement rectiligne uniforme}
\paragraph{Définition}
Pour un point en \textit{mouvement rectiligne uniforme}
\begin{itemize}
 \item la somme des forces extérieures est nulle
 \item le vecteur vitesse reste constante $$\overrightarrow{V} = \overrightarrow{Constant}$$
\end{itemize}
Voir figure \ref{fig:principe_inertie_uniforme}.
\begin{figure}[h!]
  \begin{center}
      \includegraphics[width=\columnwidth]{4.0-mvt-interactions/principe_inertie_uniforme.pdf}
  \end{center}
  \caption{Un objet sur lequel les forces se compensent peut avoir un mouvement rectiligne et uniforme. Un objet en mouvement rectiligne et uniforme est soumis à des forces qui doivent se compenser.}
  \label{fig:principe_inertie_uniforme}
\end{figure}

\subsection{Cas du point en chute libre à une dimension}
\paragraph{Définition}
Pour un point en \textit{mouvement de chute libre à une dimension}
\begin{itemize}
 \item la somme des forces extérieures est non nulle
 \item le vecteur vitesse varie linéairement avec le temps $\overrightarrow{V} = (a \times t + b)\overrightarrow{i}$
\end{itemize}
Voir figure \ref{fig:principe_inertie_accelere}.
\begin{figure}[h!]
  \begin{center}
      \includegraphics[width=\columnwidth]{4.0-mvt-interactions/principe_inertie_accelere.pdf}
  \end{center}
  \caption{Un objet en chute libre sans frottement a un mouvement rectiligne accéléré. Les forces ne se compensent pas, le vecteur vitesse change.}
  \label{fig:principe_inertie_accelere}
\end{figure}

