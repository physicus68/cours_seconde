\chapter{Ondes et signaux}
\intro{ Pour communiquer l'Être Humain émet et perçois différents types d'ondes qui pourront transporter des informations: les ondes sonores et les ondes électromagnétiques, dont font partie les ondes lumineuses. La lumière apporte ainsi des informations sur les objets qui nous entourent et en physique, l'optique et la spectroscopie sont les disciplines qui étudient la lumière. Les humains utilisent aussi l'électricité pour transporter de l'énergie et échanger des informations.
}

%%%%%%%%%%%%%%%%%%%%%%%%%%%%%%%%%%%%%%%%%%%%%%%%%%%%%%%%%%%%%%%%%%%%%
\section{Émission et perception d'un son}
\subsection{Émission et propagation d'un signal sonore}
\paragraph{Définition} Le \textit{son} est une \textit{perturbation de la pression}
dans un milieu qui se propage de proche en proche. Pour \textit{créer un son}, il faut donc créer une surpression, en utilisant par exemple un \textit{objet qui vibre} comme un diapason ou un fil raide tendu. Voir figure \ref{fig:objets_vibrants}.
\begin{figure}[h!]
  \begin{center}
      \includegraphics[width=\columnwidth]{5.0-ondes-signaux/objets_vibrants.pdf}
  \end{center}
  \caption{Un diapason ou une corde en acier tendue peuvent produire une onde de pression (son) lors de leur mouvement de vibration.}
  \label{fig:objets_vibrants}
\end{figure}

\paragraph{Définition} Pour que la vibration soit bien transmisse à l'air, on peut
utiliser une \textit{caisse de résonance}.
\paragraph{Exemple} Pour bien rendre audible le son émis par un diapason, on le
pose sur une surface dure (table en bois) ou sur une boite creuse. Un instrument à corde (violon, guitare) utilise une caisse en bois qui amplifie le son par résonance. Voir figure \ref{fig:caisse-resonance}.
On peut aussi faire cette démonstration en plaquant un téléphone portable émettant 
un son contre une boite de conserve en tôle.
\begin{figure}[h!]
  \begin{center}
      \includegraphics[width=\columnwidth]{5.0-ondes-signaux/caisse-resonance.pdf}
  \end{center}
  \caption{Le son du diapason est amplifié grâce à une caisse de résonance. La corde métallique d'un instrument de musique est fixée sur une caisse en bois pour que le son émis soit plus fort.}
  \label{fig:caisse-resonance}
\end{figure}

\paragraph{Définition} Pour que  \textit{le son puisse se propager}, il faut la présence d'un
milieu matériel (un gaz, un liquide, un solide). Dans le vide, le son ne peut pas
se propager.
\paragraph{Exemple} Si on place un haut parleur sous une cloche à vide, le son n'est plus audible dès que l'air est pompé dans la cloche, car il n'y a plus de matière capable de transmettre les ondes de pression. Voir figure \ref{fig:propagation-son}.
\begin{figure}[h!]
  \begin{center}
      \includegraphics[width=\columnwidth]{5.0-ondes-signaux/propagation-son.pdf}
  \end{center}
  \caption{La musique émise par un appareil sous la cloche à vide est perceptible tant que l'air est présent dans la cloche. Dès le pompage mis en route, ce son est de moins en moins audible car l'air ne le transmet plus.}
  \label{fig:propagation-son}
\end{figure}

\subsection{Vitesse de propagation du son}
\paragraph{Définition} Dans \textit{l'air qui nous entour}, la vitesse du son est d'environ
 $343~m.s^{-1}$. Cette vitesse varie légèrement en fonction de la température, de l'humidité et de la pression atmosphérique.
\paragraph{Exemple} La vitesse du son dans l'air $343~m.s^{-1}$ correspond à une
vitesse de $ 1200~km.h^{-1}$, la vitesse de certains avions militaires. 
\paragraph{Exemple} Pendant un orage, on peut facilement remarquer un décalage entre le flash de l'éclair dû à la foudre et l'arrivée du son du tonnerre. La lumière se propage à $3.00 \times 10^8 ~m.s^{-1}$, sa perception est quasi instantanée, alors que le son va mettre une seconde pour parcourir $343$ mètres. En comptant les secondes de décalage et en multipliant par $343$, on a la distance entre l'éclair et nos oreilles.


\subsection{Signal sonore périodique}
\paragraph{Définition} Un \textit{signal sonore est périodique} quand il se répète identique à lui même au bout d'une durée $T$ appelée période, elle s'exprime en seconde ($s$). Voir figure \ref{fig:signal-periodique}
\begin{figure}[h!]
  \begin{center}
      \includegraphics[width=\columnwidth]{5.0-ondes-signaux/signal-periodique.pdf}
  \end{center}
  \caption{Un signal périodique possède un motif qui se répète à une intervalle de temps régulier, appelé la période du signal. Cette durée s'exprime en seconde.}
  \label{fig:signal-periodique}
\end{figure}

\paragraph{Définition} La \textit{fréquence d'un son $f$ }est le nombre de fois que le
signal se répète à l'identique par seconde, elle s'exprime en Hertz ($Hz$). Elle 
est l'inverse de la période $T$ $$f = \frac{1}{T}$$

\paragraph{Exemple} Un son de période $T=1.4~ms$ correspond à un signal sonore de fréquence $f=\frac{1}{T} = \frac{1}{1.4 \times 10^{-3}} = 714~Hz$. Une onde radio de fréquence $f=108~MHz$ correspond à un signal périodique de période $T= \frac{1}{f} =\frac{1}{108 \times 10^6} =9.3 \times 10^{-9} = 9.3~ns$.

\subsection{perception du son}
\paragraph{Définition} Le spectre audible par un être humain s'étend de $20~Hz$ à $20~kHz$.
Les fréquences inférieures à $20~Hz$ correspondent aux \textit{infra sons}, les fréquences supérieures à $20~kHz$ aux \textit{ultra sons}. Voir figure \ref{fig:spectre-audible}.
\begin{figure}[h!]
  \begin{center}
      \includegraphics[width=\columnwidth]{5.0-ondes-signaux/spectre-audible.pdf}
  \end{center}
  \caption{Chez l'être humain, le spectre audible s'étend d'environ $20~Hz$ à $20~kHz$. Certains animaux ont des spectres auditifs allant des infrasons aux ultrasons.}
  \label{fig:spectre-audible}
\end{figure}

\paragraph{Définition} En acoustique, la \textit{hauteur} d'un son correspond à sa fréquence $f$: un son haut a une fréquence élevée, un son bas a une fréquence faible.

\paragraph{Définition} En acoustique, la forme du signal périodique va donner un timbre différent à des sons ayant même hauteur (ou fréquence).

\paragraph{Définition} L'onde de pression d'un son transporte une certaine puissance, proportionnelle au carré de cette pression. L'intensité du son correspond à la puissance reçue sur une surface de $1~m^2$.

\paragraph{Définition} Le \textit{niveau d'intensité sonore} est une échelle \textit{non linéaire} qui permet de comparer la puissance d'un son par rapport à la puissance d'un son à peine audible. Quand la puissance d'un son est multipliée par $10$, le niveau d'intensité sonore augmente de $10~dB~SPL$ (Décibel sound pressure level). Voir figure \ref{fig:echelle-bruit}.
\begin{figure}[h!]
  \begin{center}
      \includegraphics[width=\columnwidth]{5.0-ondes-signaux/echelle-bruit.pdf}
  \end{center}
  \caption{L'échelle de niveaux sonores est exprimée en décibel acoustiques, c'est une échelle non linéaire, si on augmente le niveau de $10~dB$, on multiplie par $10$ la puissance acoustique émise.}
  \label{fig:echelle-bruit}
\end{figure}

\paragraph{Définition} Le niveau d'intensité sonore doit être contrôlé dans l'environnement des humains car il peut induire à terme des risques de surdité totale ou partielle. C'est également un facteur accidentogène à cause de la fatigue induite par un environnement bruyant.

\paragraph{Exemple} Vous pouvez visiter le site de l'INRS (\textit{Institut national de recherche et de sécurité pour la prévention des accidents du travail et de maladies professionnelles})
\footnote{\url{http://www.inrs.fr/risques/bruit/ce-qu-il-faut-retenir.html}} qui explique la réglementation française concernant les niveaux d'exposition au bruit . 

\section{Vision et image}
\subsection{Propagation rectiligne de la lumière}
\paragraph{Définition} La lumière se propage \textit{en ligne droite dans un milieu homogène}. Dans l'air ou dans le vide, sa vitesse de propagation est $c=3.00 \times 10^8~m.s^{-1}$.
\subsection{Spectre de la lumière}
\paragraph{Définition} Un objet chauffé à haute température va émettre de la lumière dont le spectre est continu. Voir figures \ref{fig:exemple-corps-noir}  et \ref{fig:dessin-spectre-lum-jour}.
\begin{figure}[h!]
  \begin{center}
      \includegraphics[width=\columnwidth]{5.0-ondes-signaux/exemple-corps-noir.pdf}
  \end{center}
  \caption{Un morceau d'acier chauffé à haute température va émettre de la lumière.}
  \label{fig:exemple-corps-noir}
\end{figure}
\paragraph{Définition} Un gaz à basse pression qui subit une excitation va émettre de la lumière dont le spectre est discontinu et présente des raies lumineuses spécifiques de l'élément chimique constituant le gaz. Voir figure \ref{fig:dessin-spectre-tube-fluo}.
\begin{figure}[h!]
  \begin{center}
      \includegraphics[width=\columnwidth]{5.0-ondes-signaux/dessin-spectre-tube-fluo.pdf}
  \end{center}
  \caption{Spectre visible de la lumière émise par un tube fluorescent éclairant la salle de classe. On observe la présence de raies d'émissions lumineuses dont certaines sont caractéristiques des vapeurs du mercure.}
  \label{fig:dessin-spectre-tube-fluo}
\end{figure}
\paragraph{Définition} Pour identifier une couleur dans le spectre visible, on mesure la longueur d'onde $\lambda$ du rayonnement lumineux. Pour le spectre visible, $\lambda$ varie de $400~nm$ à $700~nm$ environ. Voir figure \ref{fig:dessin-spectre-lum-jour}.
\begin{figure}[h!]
  \begin{center}
      \includegraphics[width=\columnwidth]{5.0-ondes-signaux/dessin-spectre-lum-jour.pdf}
  \end{center}
  \caption{Spectre visible de la lumière du jour, c'est un spectre continu, avec un maximum d'émission dans le vert typique d'un corps noir à $5700~{}^oC$, le Soleil. On observe aussi la présence de raies sombres, les raies d’absorption de Fraunhofer, caractéristiques de la présence de certains éléments dans le Soleil et l'atmosphère terrestre.}
  \label{fig:dessin-spectre-lum-jour}
\end{figure}

\subsection{Réflexion et réfraction}
\paragraph{Définition} La réflexion d'un rayon lumineux sur une surface se fait avec le même angle que l'angle d'incidence. Voir figure \ref{fig:loi-reflexion-optique}.
\begin{figure}[h!]
  \begin{center}
      \includegraphics[width=\columnwidth]{5.0-ondes-signaux/loi-reflexion-optique.pdf}
  \end{center}
  \caption{L'angle de réflexion et l'angle d'incidence sont identiques. Ils se mesurent par rapport à la normale à la surface réfléchissante.}
  \label{fig:loi-reflexion-optique}
\end{figure}

\paragraph{Définition} Quand un rayon lumineux traverse une surface qui sépare deux milieux différents, il y a un changement de direction du rayon lumineux, c'est le phénomène de diffraction de la lumière. Il est décrit par la loi de Snell-Descartes $$n_1 \times \sin{ i} = n_2 \times \sin{ r}$$ Voir figure \ref{fig:loi-refraction-optique}.
\begin{figure}[h!]
  \begin{center}
      \includegraphics[width=\columnwidth]{5.0-ondes-signaux/loi-refraction-optique.pdf}
  \end{center}
  \caption{La réfraction est le changement de direction de propagation d'un rayon lumineux quand il change de milieu. Les angles d'incidence $i$ et de réfraction $r$ sont liés aux indices optiques $n_1$ et $n_2$ des deux milieux grâce à la loi de \textit{Snell-Descartes}.}
  \label{fig:loi-refraction-optique}
\end{figure}

\paragraph{Exemple}

\subsection{Dispersion de la lumière}
\paragraph{Définition} Un prisme est un objet en verre qui utilise le phénomène de réfraction de la lumière pour décomposer un mélange de radiation lumineuses, car le phénomène dépend de la longueur d'onde de la lumière traversant le prisme. Voir figure \ref{fig:prisme-optique}.
\begin{figure}[h!]
  \begin{center}
      \includegraphics[width=\columnwidth]{5.0-ondes-signaux/prisme-optique.pdf}
  \end{center}
  \caption{Un prisme décompose la lumière en déviant les rayons lumineux selon leur couleur. La lumière bleue est plus déviée que la lumière rouge. La déviation se fait du coté de la base du prisme. L'ordre des couleurs est celui de l'arc en ciel.}
  \label{fig:prisme-optique}
\end{figure}

\paragraph{Définition} La spectroscopie permet d'avoir des informations notamment  sur la composition chimique de la source émettrice de lumière. C'est une technique d'analyse fondamentale en science.



\subsection{Lentille mince convergente}
\paragraph{Définition} Une lentille mince convergente est caractérisée par sa distance focale $f$ qui permet de définir la position des foyers images $F'$ et foyer objet $F$. Voir figure \ref{fig:lentille-mince-convergente}.
\begin{figure}[h!]
  \begin{center}
      \includegraphics[width=\columnwidth]{5.0-ondes-signaux/lentille-mince-convergente.pdf}
  \end{center}
  \caption{Une lentille mince convergente est un objet qui permet de concentrer les rayons lumineux. Elle est caractérisée par son axe optique, sa distance focale $f'$ qui permet de placer ses foyers objet $F$ et image $F'$ par rapport à son centre optique $O$.}
  \label{fig:lentille-mince-convergente}
\end{figure}


\paragraph{Définition} Une lentille mince peut former une image réelle d'un objet en la projetant sur un écran, voir figure \ref{fig:lentille-grandissement}. Cette image sera inversée et plus ou moins grande. On précisera alors le grandissement $$ G = \frac{\mbox{taille image}}{\mbox{taille objet}}$$ 

\begin{figure}[h!]
  \begin{center}
      \includegraphics[width=\columnwidth]{5.0-ondes-signaux/lentille-grandissement.pdf}
  \end{center}
  \caption{Le grandissement est le rapport entre la taille de l'image et la taille de l'objet. Le grandissement est négatif si l'image est renversée.}
  \label{fig:lentille-grandissement}
\end{figure}

\paragraph{Définition} On peut prédire la position de l'image connaissant les caractéristiques géométriques de la lentille et la position de l'objet grâce à une construction géométrique. Voir figure \ref{fig:lentille-construction-image}.
\begin{figure}[h!]
  \begin{center}
      \includegraphics[width=\columnwidth]{5.0-ondes-signaux/lentille-construction-image.pdf}
  \end{center}
  \caption{On considère un point $B$ sur un objet dont on veut construire l'image (a). Depuis ce point, on trace un rayon parallèle à l'axe optique qui passe ensuite par le foyer image après traversée de la lentille (b). Puis on trace un rayon passant par le centre optique $O$ qui n'est pas dévié (c). On trace un rayon passant par le foyer objet $F$ et qui sort de la lentille parallèle à l'axe optique (d). Ces trois rayons se croisent en un point $B'$ image de $B$ (e). Tous les rayons quittant le point objet $B$ et traversant la lentille convergeront vers le point image $B'$, voir la zone grise sur (f).}
  \label{fig:lentille-construction-image}
\end{figure}



\subsection{Modèle simplifié de l’œil}
\paragraph{Définition} On modélise le fonctionnement optique de l’œil comme étant une lentille de focale variable qui projette l'image d'un objet sur la rétine, il est équipé d'un diaphragme(l'iris) qui laisse entrer plus ou moins de lumière pour éviter l'éblouissement. L’œil ne se déforme pas, pour accommoder, c'est le cristallin qui se déforme pour modifier sa focale. Voir figure \ref{fig:oeil-simplifie}.
\begin{figure}[h!]
  \begin{center}
      \includegraphics[width=\columnwidth]{5.0-ondes-signaux/oeil-simplifie.pdf}
  \end{center}
  \caption{L’œil simplifié est composé d'un iris qui laisse plus ou moins la lumière entrer dans l’œil, puis d'un cristallin qui est une lentille déformable, elle a une focale variable qui permet l’accommodation, et d'une rétine couverte de cellules nerveuses sensibles à la lumière (cônes et bâtonnets) sur laquelle se projette l'image des objets observés. }
  \label{fig:oeil-simplifie}
\end{figure}



\section{Signaux et capteurs}
\subsection{Circuit électrique}
\paragraph{Circuit} Un circuit électrique contient en général
\begin{itemize}
 \item un ou plusieurs générateurs de courant ou de tension
 \item des récepteurs (lampe, moteur électrique, appareil électronique)
 \item ces dispositifs sont obligatoirement reliés par au moins deux câbles électriques.
\end{itemize}
Voir figure \ref{fig:exemple-circuit-electrique}.
\begin{figure}[h!]
  \begin{center}
      \includegraphics[width=\columnwidth]{5.0-ondes-signaux/exemple-circuit-electrique.pdf}
  \end{center}
  \caption{Ce circuit électrique se compose d'une source de tension (1), qui alimente un moteur (4) quand on ferme l'interrupteur (2) . Quand le moteur est sous tension, une diode luminescente s'allume (5) et le courant dans cette diode est limité par une résistance électrique (3). }
  \label{fig:exemple-circuit-electrique}
\end{figure}


\paragraph{Potentiel électrique}
La différence de potentiel entre deux bornes du générateur va mettre en mouvement les charges électriques présentes dans le circuit 
\begin{itemize}
 \item de la borne $+$ vers la borne $-$ pour les charges positives
 \item de la borne $-$ vers la borne $+$ pour les charges négatives
\end{itemize}
En traversant un dipôle du circuit, il y a une perte de potentiel, qui est plus faible à la sortie du dipôle.
On symbolise la différence de potentiel $U_{AB}=V_A-V_b$ à l'aide d'une flèche dessinée à coté du dipôle. Voir figure \ref{fig:potentiel-electrique}.
\begin{figure}[h!]
  \begin{center}
      \includegraphics[width=\columnwidth]{5.0-ondes-signaux/potentiel-electrique.pdf}
  \end{center}
  \caption{Aux bornes $A$ et $B$ d'un dipôle, les potentiels $V_A$ et $V_B$ peuvent être différents et il y a une \textit{différence de potentiel} $U_{AB} = V_A - V_B$ schématisée par une flèche à coté du dipôle. Le courant électrique $I$ circule de la borne $+$ du générateur vers la borne $-$. Il représente le déplacement des charges électriques positives.}
  \label{fig:potentiel-electrique}
\end{figure}

\paragraph{Courant électrique} Le courant électrique est le déplacement des charges électriques positives et négatives dans le circuit. Par convention, on indique seulement le déplacement des charges positives dans le circuit à l'aide d'une flèche dessinée sur une branche du circuit. Voir figure \ref{fig:potentiel-electrique}.
\subsection{Mesures électriques}
\paragraph{Intensité du courant} L'intensité du courant se mesure en Ampère ($A$) à l'aide d'un ampèremètre qui doit être traversé par le courant. Les contrôleurs universels possèdent ainsi deux bornes, l'une pour l'entrée du courant (rouge, avec le symbole A) et une pour la sortie du courant (noire) avec le symbole COM. Pour placer un ampèremètre dans un circuit, il faut ouvrir le circuit pour brancher cet appareil, cela doit se faire avec l'alimentation électrique coupée. Voir figure \ref{fig:mesures-electriques}.
\paragraph{Différence de potentiel} La différence de potentiel se mesure en Volt ($V$) à l'aide d'un voltmètre muni de deux câbles, l'un relié à une borne de référence (COM), l'autre relié à la borne V (rouge). On mesure la différence de potentiel entre ces deux bornes. Voir figure \ref{fig:mesures-electriques}.
\begin{figure}[h!]
  \begin{center}
      \includegraphics[width=\columnwidth]{5.0-ondes-signaux/mesures-electriques.pdf}
  \end{center}
  \caption{Un contrôleur universel mesure l'intensité $I$ du courant électrique, un autre contrôleur universel mesure la différence de potentiel $U_{AB}$ au borne d'un dipôle.}
  \label{fig:mesures-electriques}
\end{figure}


\subsection{Loi des nœuds}
\paragraph{Définition} Dans un circuit électrique, le courant doit se conserver, tous les courants qui entrent dans un nœud doivent être égaux à tous les courants qui quittent ce nœud.
\paragraph{Exemples} Dans le circuit de la figure \ref{fig:loi-des-noeuds}, d'après la loi des nœuds, on pourra écrire 
\begin{itemize}
 \item au nœud $A$, on a $I_1 = I_2 + I_3$
 \item au nœud $C$,on a $0 = I_5 + I_6 + I_7$
 \item au nœud $B$,on a $I_2 = I_4$
 \item dans la branche $ABC$,on a $I_6 = - I_4$ 
 \item dans la branche du générateur,on a $I_1 = I_7$ 
\end{itemize}

\begin{figure}[h!]
  \begin{center}
      \includegraphics[width=\columnwidth]{5.0-ondes-signaux/loi-des-noeuds.pdf}
  \end{center}
  \caption{La loi des nœuds signifie que le courant doit se conserver dans un circuit électrique.}
  \label{fig:loi-des-noeuds}
\end{figure}
\subsection{Loi des mailles}
\paragraph{Définition} Dans une maille (ou boucle) d'un circuit électrique, après avoir défini arbitrairement un sens de parcours de la boucle, la somme des différences de potentiel est nulle. On comptera positivement une différence de potentielle dont la flèche est dans le sens du parcours, négativement dans le cas contraire.
\paragraph{Exemples} Dans le circuit de la figure \ref{fig:loi-des-mailles}, en choisissant de tourner dans les boucles dans le sens des aiguilles d'une montre, d'après la loi des mailles, on pourra écrire 
\begin{itemize}
 \item $U_{AD} + U_{CA} - U_{CD} = 0~ V$
 \item $U_{AD} - U_{AC} - U_{BC} - U_{CD}  = 0~ V$
 \item $-U_{AC} - U_{BC} - U_{CA} = 0~ V$
\end{itemize}
\begin{figure}[h!]
  \begin{center}
      \includegraphics[width=\columnwidth]{5.0-ondes-signaux/loi-des-mailles.pdf}
  \end{center}
  \caption{La loi des mailles signifie que la somme des différences de potentiels dans une maille est nulle.}
  \label{fig:loi-des-mailles}
\end{figure}

\subsection{Caractéristique tension $U$ et courant $I$ d'un dipôle}
 \paragraph{Définition}La caractéristique courant-tension d'un dipôle est un graphique expérimental qui représente la relation entre l'intensité du courant traversant le dipôle et la tension au borne de ce dipôle. Voir schéma \ref{fig:caracteristique_UI_diode}.
\begin{figure}[h!]
  \begin{center}
      \includegraphics[width=\columnwidth]{5.0-ondes-signaux/caracteristique_UI_diode.pdf}
  \end{center}
  \caption{Caractéristique courant-tension d'une diode. C'est un dipôle qui ne laisse circuler le courant que dans un seul sens. Il peut émettre de la lumière (LED) ou la détecter (photodiode).}
  \label{fig:caracteristique_UI_diode}
\end{figure}
\paragraph{Définition} Si on dispose de la caractéristique courant-tension d'un dipôle, on peut trouver son point de fonctionnement, c'est à dire la valeur de $I$ qui le traversera pour une tension $U$ à ses bornes.
\paragraph{Définition} La caractéristique d'un dipôle ohmique est $$U_{AB}= R\times I$$ avec \begin{itemize}
   \item $U_{AB}$ en volt ($V$)
   \item $I$ en ampère ($A$), orienté de $A$ vers $B$ 
   \item $R$ en Ohms $\Omega$
\end{itemize}

\begin{figure}[h!]
  \begin{center}
      \includegraphics[width=\columnwidth]{5.0-ondes-signaux/caracteristique_UI_resistance.pdf}
  \end{center}
  \caption{Caractéristique courant-tension d'une résistance, la tension $U_{AB}$ est proportionnelle au courant $I$ par le facteur $R$.  On ne peut pas cependant avoir n'importe quelle valeur pour les tensions et les courants, la résistance chauffe et peut être détruite.}
  \label{fig:caracteristique_UI_resistance}
\end{figure}


\subsection{Capteurs électriques}
\paragraph{Définition}Un capteur électrique fournit une tension électrique ou un courant dont la valeur dépend d'une grandeur physique que l'on cherche à mesurer, comme la température, la pression, l'intensité lumineuse. On retrouve ces capteurs dans des thermostats, des détecteurs de lumière, des jauges de déformation (sciences de l'ingénieur), etc. ...
\paragraph{Exemples}
\begin{itemize}
 \item La photorésistance (LDR \textit{ light-dependent resistor}) a une résistance électrique qui décroît avec l'augmentation de l'intensité lumineuse reçue.
 \item La thermistance a une résistance électrique qui décroît ou croît avec l'augmentation de la température du composant (il existe deux types de thermistances).
 \item La sonde au platine (PT100 ou PT1000) a une résistance électrique qui dépend de la température du composant. 
\end{itemize}
\begin{figure}[h!]
  \begin{center}
      \includegraphics[width=\columnwidth]{5.0-ondes-signaux/symboles-capteurs.pdf}
  \end{center}
  \caption{Symboles électriques de différents capteurs résistifs.}
  \label{fig:symboles-capteurs}
\end{figure}



