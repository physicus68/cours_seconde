\chapter{Annexes}
%%%%%%%%%%%%%%%%%%%%%%%%%%%%%%%%%%%%%%%%%%%%%%%%%%%%%%%%%%%%%%%%%%%%%
\section{Table de la classification périodique des éléments}
\paragraph{Classification périodique} La table de la classification périodique des éléments est donnée sur la table \ref{table_IUPAC_classification}.
\paragraph{Traduction du nom des éléments} Les éléments des trois premières lignes se traduisent ainsi:

\begin{table}[h!]
  \centering
  \begin{tabu} to 0.95\linewidth {  X[2,l]  X[l]  }
    \hline
      \textbf{Français} & \textbf{Anglais} \\				      
    \hline
      Hydrogène & Hydrogen \\  
      Hélium 	& Helium \\
      Lithium	& Lithium \\
      Béryllium & Beryllium \\
      Bore	& Boron \\
      Carbone	& Carbon \\
      Azote	& Nitrogen \\
      Oxygène	& Oxygen \\
      Fluor 	& Fluorine \\
      Néon	& Neon \\
      Sodium 	& Sodium\\
      Magnésium & Magnesium\\
      Aluminium & Aluminum \\
      Silicium 	& Silicon \\
      Phosphore & Phosphorus \\
      Souffre	& Sulfur \\
      Chlore	& Chlorine \\
      Argon	& Argon \\
      ~ & ~\\
      Fer	& Iron \\
      Cuivre	& Copper \\
      Brome	& Bromine \\
      Argent	& Silver \\
      Étain	& Tin \\
      Platine	& Platinum\\
      Or	& Gold \\
      Mercure 	& Mercury \\
      Plomb	& Lead \\
    \hline
  \end{tabu}
  \caption{Traduction français anglais du nom de quelques éléments}
  \label{tab:traduction-nom-elements}
\end{table}

\begin{figure*}[h]
  \begin{center}
      \includegraphics[width=1.4\textwidth, angle=90]{IUPAC_Periodic_Table-01Dec18_modif.pdf}
  \end{center}  
  \caption{Table de la classification périodique des éléments}
  \label{table_IUPAC_classification}
\end{figure*}
